\documentclass{manual}
\usepackage[T1]{fontenc}

\title{ceODBC}

\author{Anthony Tuininga}
\authoraddress{
        \strong{Colt Engineering}\\
        Email: \email{anthony.tuininga@gmail.com}
}

\date{\today}                   % date of release
\release{1.0}                   % software release
\setreleaseinfo{}               % empty for final release
\setshortversion{1.0}           % major.minor only for software

\begin{document}

\maketitle

\ifhtml
\chapter*{Front Matter\label{front}}
\fi

Copyright \copyright{} 2007 Colt Engineering.
All rights reserved.

See the end of this document for complete license and permissions
information.

\begin{abstract}

\noindent
ceODBC is a Python extension module that enables access to databases using the
ODBC API and conforms to the Python database API 2.0 specifications with a few
exceptions.
See \url{http://www.python.org/topics/database/DatabaseAPI-2.0.html} for more
information on the Python database API specification.

\end{abstract}

\tableofcontents

\chapter{Module Interface\label{module}}

\begin{funcdesc}{Binary}{\var{string}}
  Construct an object holding a binary (long) string value.
\end{funcdesc}

\begin{funcdesc}{Connection}{\var{dsn}}
\funcline{connect}{\var{dsn}}
  Constructor for creating a connection to the database. Return a Connection
  object (\ref{connobj}). The only argument expected is the DSN in the format
  that ODBC expects.
\end{funcdesc}

\begin{funcdesc}{Cursor}{\var{connection}}
  Constructor for creating a cursor.  Return a new Cursor object
  (\ref{cursorobj}) using the connection.

  \strong{NOTE:} This method is an extension to the DB API definition.
\end{funcdesc}

\begin{funcdesc}{Date}{\var{year}, \var{month}, \var{day}}
  Construct an object holding a date value.
\end{funcdesc}

\begin{funcdesc}{DateFromTicks}{\var{ticks}}
  Construct an object holding a date value from the given ticks value (number
  of seconds since the epoch; see the documentation of the standard Python
  time module for details).
\end{funcdesc}

\begin{funcdesc}{Time}{\var{hour}, \var{minute}, \var{second}}
  Construct an object holding a time value.
\end{funcdesc}

\begin{funcdesc}{TimeFromTicks}{\var{ticks}}
  Construct an object holding a time value from the given ticks value (number
  of seconds since the epoch; see the documentation of the standard Python
  time module for details).
\end{funcdesc}

\begin{funcdesc}{Timestamp}{\var{year}, \var{month}, \var{day},
    \var{hour}, \var{minute}, \var{second}}
  Construct an object holding a time stamp value.
\end{funcdesc}

\begin{funcdesc}{TimestampFromTicks}{\var{ticks}}
  Construct an object holding a time stamp value from the given ticks value
  (number of seconds since the epoch; see the documentation of the standard
  Python time module for details).
\end{funcdesc}

\section{Constants}\label{constants}

\begin{datadesc}{apilevel}
  String constant stating the supported DB API level. Currently '2.0'.
\end{datadesc}

\begin{datadesc}{buildtime}
  String constant stating the time when the binary was built.

  \strong{NOTE:} This attribute is an extension to the DB API definition.
\end{datadesc}

\begin{datadesc}{BINARY}
  This type object is used to describe columns in a database that are binary.
\end{datadesc}

\begin{datadesc}{DATETIME}
  This type object is used to describe columns in a database that are dates.
\end{datadesc}

\begin{datadesc}{NUMBER}
  This type object is used to describe columns in a database that are numbers.
\end{datadesc}

\begin{datadesc}{paramstyle}
  String constant stating the type of parameter marker formatting expected by
  the interface. Currently 'qmark' as in 'where name = ?'.
\end{datadesc}

\begin{datadesc}{ROWID}
  This type object is used to describe the pseudo column "rowid".
\end{datadesc}

\begin{datadesc}{STRING}
  This type object is used to describe columns in a database that are strings.
\end{datadesc}

\begin{datadesc}{threadsafety}
  Integer constant stating the level of thread safety that the interface
  supports. Currently 2, which means that threads may share the module and
  connections, but not cursors. Sharing means that a thread may use a resource
  without wrapping it using a mutex semaphore to implement resource locking.
\end{datadesc}

\begin{datadesc}{version}
  String constant stating the version of the module. Currently '\version{}'.

  \strong{NOTE:} This attribute is an extension to the DB API definition.
\end{datadesc}

\section{Exceptions}\label{exceptions}

\begin{datadesc}{Warning}
  Exception raised for important warnings and defined by the DB API but not
  actually used by ceODBC.
\end{datadesc}

\begin{datadesc}{Error}
  Exception that is the base class of all other exceptions defined by
  ceODBC and is a subclass of the Python StandardError exception (defined in
  the module exceptions).
\end{datadesc}

\begin{datadesc}{InterfaceError}
  Exception raised for errors that are related to the database interface rather
  than the database itself. It is a subclass of Error.
\end{datadesc}

\begin{datadesc}{DatabaseError}
  Exception raised for errors that are related to the database. It is a
  subclass of Error.
\end{datadesc}

\begin{datadesc}{DataError}
  Exception raised for errors that are due to problems with the processed data.
  It is a subclass of DatabaseError.
\end{datadesc}

\begin{datadesc}{OperationalError}
  Exception raised for errors that are related to the operation of the database
  but are not necessarily under the control of the progammer. It is a
  subclass of DatabaseError.
\end{datadesc}

\begin{datadesc}{IntegrityError}
  Exception raised when the relational integrity of the database is affected.
  It is a subclass of DatabaseError.
\end{datadesc}
 
\begin{datadesc}{InternalError}
  Exception raised when the database encounters an internal error.
  It is a subclass of DatabaseError.
\end{datadesc}
 
\begin{datadesc}{ProgrammingError}
  Exception raised for programming errors. It is a subclass of DatabaseError.
\end{datadesc}
 
\begin{datadesc}{NotSupportedError}
  Exception raised when a method or database API was used which is not
  supported by the database. It is a subclass of DatabaseError.
\end{datadesc}
 
\section{Variable Types}\label{vartypes}

\strong{NOTE:} The DB API definition does not define these object.

These classes all create variable objects (\ref{varobj}). They are created
implicitly by cursor.execute() as needed and normally need not be created
directly. These classes can also be passed in to cursor.setinputsizes() in
favor of the types defined by the DB API in order to have finer control over
the types of variables created.

\begin{datadesc}{BigIntegerVar}
  Variable used to bind and/or fetch big integers. Values are returned as
  Python longs and accept Python integers or longs.
\end{datadesc}

\begin{datadesc}{BinaryVar}
  Variable used to bind and/or fetch binary data. Values are returned as Python
  buffer objects and accept Python objects that implement the buffer protocol.
\end{datadesc}

\begin{datadesc}{BitVar}
  Variable used to bind and/or fetch bits. Values are returned as Python
  booleans and accept the same.
\end{datadesc}

\begin{datadesc}{DateVar}
  Variable used to bind and/or fetch dates. Values are returned as Python
  datetime.date objects and accept Python datetime.date or datetime.datetime
  objects.
\end{datadesc}

\begin{datadesc}{DecimalVar}
  Variable used to bind and/or fetch decimal numbers. Values are returned as
  Python decimal.Decimal objects and accept the same.
\end{datadesc}

\begin{datadesc}{DoubleVar}
  Variable used to bind and/or fetch floating point numbers. Values are
  returned as Python floats and accept Python integers or floats.
\end{datadesc}

\begin{datadesc}{IntegerVar}
  Variable used to bind and/or fetch integers. Values are returned as Python
  integers and accept the same.
\end{datadesc}

\begin{datadesc}{LongBinaryVar}
  Variable used to bind and/or fetch long binary data. Values are returned as
  Python buffer objects and accept Python objects that implement the buffer
  protocol.
\end{datadesc}

\begin{datadesc}{LongStringVar}
  Variable used to bind and/or fetch long string data. Values are returned as
  Python strings and accept the same.
\end{datadesc}

\begin{datadesc}{StringVar}
  Variable used to bind and/or fetch string data. Values are returned as Python
  strings and accept the same.
\end{datadesc}

\begin{datadesc}{TimestampVar}
  Variable used to bind and/or fetch timestamps. Values are returned as Python
  datetime.datetime objects and accept Python datetime.date or
  datetime.datetime objects.
\end{datadesc}

\chapter{Connection Objects\label{connobj}}

\strong{NOTE}: Any outstanding changes will be rolled back when the connection
object is destroyed or closed.

\begin{funcdesc}{close}{}
  Close the connection now, rather than whenever __del__ is called. The
  connection will be unusable from this point forward; an Error exception will
  be raised if any operation is attempted with the connection. The same applies
  to any cursor objects trying to use the connection.
\end{funcdesc}

\begin{funcdesc}{commit}{}
  Commit any pending transactions to the database.
\end{funcdesc}

\begin{funcdesc}{cursor}{}
  Return a new Cursor object (\ref{cursorobj}) using the connection.
\end{funcdesc}

\begin{datadesc}{dsn}
  This read-only attribute returns the DSN of the database to which a
  connection has been established.

  \strong{NOTE:} This attribute is an extension to the DB API definition.
\end{datadesc}

\begin{funcdesc}{rollback}{}
  Rollback any pending transactions.
\end{funcdesc}

\chapter{Cursor Objects\label{cursorobj}}

\begin{datadesc}{arraysize}
  This read-write attribute specifies the number of rows to fetch at a time
  internally and is the default number of rows to fetch with the fetchmany()
  call. It defaults to 1 meaning to fetch a single row at a time. Note that
  this attribute can drastically affect the performance of a query since it
  directly affects the number of network round trips that need to be performed.
\end{datadesc}

\begin{datadesc}{bindarraysize}
  This read-write attribute specifies the number of rows to bind at a time and
  is used when creating variables via setinputsizes(). It defaults to 1 meaning
  to bind a single row at a time.

  \strong{NOTE:} The DB API definition does not define this attribute.
\end{datadesc}

\begin{funcdesc}{callfunc}{\var{name}, \var{returnType}, \var{*args}}
  Call a function with the given name. Parameters may also be passed as a
  single list or tuple to conform to the DB API. The return type is specified
  in the same notation as is required by setinputsizes(). The result of the
  call is the return value of the function.
\end{funcdesc}

\begin{funcdesc}{callproc}{\var{name}, \var{*args}}
  Call a procedure with the given name. Parameters may also be passed as a
  single list or tuple to conform to the DB API. The result of the call is a
  modified copy of the input sequence. Input parameters are left untouched;
  output and input/output parameters are replaced with possibly new values.
\end{funcdesc}

\begin{funcdesc}{close}{}
  Close the cursor now, rather than whenever __del__ is called. The cursor will
  be unusable from this point forward; an Error exception will be raised if any
  operation is attempted with the cursor.
\end{funcdesc}

\begin{datadesc}{connection}
  This read-only attribute returns a reference to the connection object on
  which the cursor was created.

  \strong{NOTE:} This attribute is an extension to the DB API definition but it
  is mentioned in PEP 249 as an optional extension.
\end{datadesc}

\begin{datadesc}{description}
  This read-only attribute is a sequence of 7-item sequences. Each of these
  sequences contains information describing one result column: (name, type,
  display_size, internal_size, precision, scale, null_ok). This attribute will
  be None for operations that do not return rows or if the cursor has not had
  an operation invoked via the execute() method yet.

  The type will be one of the variable type objects (\ref{vartypes}) and is
  comparable to the type objects defined by the DB API.
\end{datadesc}

\begin{funcdesc}{execute}{\var{statement}, {\var{*args}}}
  Execute a statement against the database. Paramters may also be passed as a
  single list or tuple to conform to the DB API.

  A reference to the statement will be retained by the cursor. If None or the
  same string object is passed in again, the cursor will execute that
  statement again without performing a prepare or rebinding and redefining.
  This is most effective for algorithms where the same statement is used, but
  different parameters are bound to it (many times).

  For maximum efficiency when reusing an statement, it is best to use the
  setinputsizes() method to specify the parameter types and sizes ahead of
  time; in particular, None is assumed to be a string of length 1 so any
  values that are later bound as numbers or dates will raise a TypeError
  exception.

  If the statement is a query, the cursor is returned as a convenience since
  cursors implement the iterator protocol and there is thus no need to call
  one of the appropriate fetch methods; otherwise None is returned.

  \strong{NOTE:} The DB API definition does not define the return value of this
  method.
\end{funcdesc}

\begin{funcdesc}{executemany}{\var{statement}, \var{parameters}}
  Prepare a statement for execution against a database and then execute it
  against all parameter sequences found in the sequence parameters. The
  statement is managed in the same way as the execute() method manages it.
\end{funcdesc}

\begin{funcdesc}{fetchall}{}
  Fetch all (remaining) rows of a query result, returning them as a list of
  tuples. An empty list is returned if no more rows are available. Note that
  the cursor's arraysize attribute can affect the performance of this
  operation, as internally reads from the database are done in batches
  corresponding to the arraysize.

  An exception is raised if the previous call to execute() did not produce any
  result set or no call was issued yet.
\end{funcdesc}

\begin{funcdesc}{fetchmany}{\optional{\var{numRows=cursor.arraysize}}}
  Fetch the next set of rows of a query result, returning a list of tuples. An
  empty list is returned if no more rows are available. Note that the cursor's
  arraysize attribute can affect the performance of this operation.

  The number of rows to fetch is specified by the parameter. If it is not
  given, the cursor's arrysize attribute determines the number of rows to be
  fetched. If the number of rows available to be fetched is fewer than the
  amount requested, fewer rows will be returned.

  An exception is raised if the previous call to execute() did not produce any
  result set or no call was issued yet.
\end{funcdesc}

\begin{funcdesc}{fetchone}{}
  Fetch the next row of a query result set, returning a single tuple or None
  when no more data is available.

  An exception is raised if the previous call to execute() did not produce any
  result set or no call was issued yet.
\end{funcdesc}

\begin{funcdesc}{__iter__}{}
  Returns the cursor itself to be used as an iterator.

  \strong{NOTE:} This method is an extension to the DB API definition but it
  is mentioned in PEP 249 as an optional extension.
\end{funcdesc}

\begin{funcdesc}{next}{}
  Fetch the next row of a query result set, using the same semantics as
  the method fetchone().

  \strong{NOTE:} This method is an extension to the DB API definition but it
  is mentioned in PEP 249 as an optional extension.
\end{funcdesc}

\begin{funcdesc}{prepare}{\var{statement}}
  This can be used before a call to execute() to define the statement that will
  be executed. When this is done, the prepare phase will not be performed when
  the call to execute() is made with None or the same string object as the
  statement.

  \strong{NOTE:} The DB API definition does not define this method.
\end{funcdesc}

\begin{datadesc}{rowcount}
  This read-only attribute specifies the number of rows that have currently
  been fetched from the cursor (for select statements) or that have been
  affected by the operation (for insert, update and delete statements).
\end{datadesc}

\begin{datadesc}{rowfactory}
  This read-write attribute specifies a method to call for each row that is
  retrieved from the database. Ordinarily a tuple is returned for each row but
  if this attribute is set, the method is called with the argument tuple that
  would normally be returned and the result of the method is returned instead.

  \strong{NOTE:} The DB API definition does not define this attribute.
\end{datadesc}

\begin{funcdesc}{setinputsizes}{\var{*args}}
  This can be used before a call to execute() to predefine memory areas for the
  operation's parameters. Each parameter should be a type object corresponding
  to the input that will be used or it should be an integer specifying the
  maximum length of a string parameter. The singleton None can be used as a
  parameter to indicate that no space should be reserved for that position.
  Note that in order to conform to the DB API, passing a single argument which
  is a list or tuple will treat that list or tuple as the arguments sequence.
\end{funcdesc}

\begin{funcdesc}{setoutputsize}{\var{size}, \optional{\var{column}}}
  This can be used before a call to execute() to predefine memory areas for the
  long columns that will be fetched. The column is specified as an index into
  the result sequence. Not specifying the column will set the default size for
  all large columns in the cursor.
\end{funcdesc}

\begin{datadesc}{statement}
  This read-only attribute provides the string object that was previously
  prepared with prepare() or executed with execute().

  \strong{NOTE:} The DB API definition does not define this attribute.
\end{datadesc}

\chapter{Variable Objects\label{varobj}}

\strong{NOTE:} The DB API definition does not define this object.

\begin{datadesc}{bufferSize}
  This read-only attribute returns the size of the buffer allocated for each
  element.
\end{datadesc}

\begin{funcdesc}{getvalue}{\optional{\var{pos=0}}}
  Return the value at the given position in the variable.
\end{funcdesc}

\begin{datadesc}{input}
  This read-write attribute specifies whether the variable is used as an input
  variable and should normally be left as True.
\end{datadesc}

\begin{datadesc}{numElements}
  This read-only attribute returns the number of elements allocated.
\end{datadesc}

\begin{datadesc}{output}
  This read-write attribute specifies whether the variable is used as an output
  variable. It should normally be left as False except when calling stored
  procedures with output variables.
\end{datadesc}

\begin{datadesc}{scale}
  This read-only attribute returns the scale of the variable.
\end{datadesc}

\begin{funcdesc}{setvalue}{\var{pos}, \var{value}}
  Set the value at the given position in the variable.
\end{funcdesc}

\begin{datadesc}{size}
  This read-only attribute returns the size of the variable.
\end{datadesc}

\chapter{License}\label{license}

\centerline{\strong{LICENSE AGREEMENT FOR ceODBC \version}}

Copyright \copyright{} 2007, Colt Engineering, Edmonton, Alberta, Canada.
All rights reserved.

Redistribution and use in source and binary forms, with or without
modification, are permitted provided that the following conditions are met:

\begin{enumerate}
\item
    Redistributions of source code must retain the above copyright notice,
    this list of conditions, and the disclaimer that follows.

\item
    Redistributions in binary form must reproduce the above copyright
    notice, this list of conditions, and the following disclaimer in the
    documentation and/or other materials provided with the distribution.

\item
    Neither the name of Colt Engineering nor the names of its contributors may
    be used to endorse or promote products derived from this software
    without specific prior written permission.
\end{enumerate}

DISCLAIMER:
THIS SOFTWARE IS PROVIDED BY COLT ENGINEERING AND CONTRIBUTORS *AS IS*
AND ANY EXPRESS OR IMPLIED WARRANTIES, INCLUDING, BUT NOT LIMITED TO,
THE IMPLIED WARRANTIES OF MERCHANTABILITY AND FITNESS FOR A
PARTICULAR PURPOSE ARE DISCLAIMED. IN NO EVENT SHALL COLT ENGINEERING
OR CONTRIBUTORS BE LIABLE FOR ANY DIRECT, INDIRECT, INCIDENTAL,
SPECIAL, EXEMPLARY, OR CONSEQUENTIAL DAMAGES (INCLUDING, BUT NOT
LIMITED TO, PROCUREMENT OF SUBSTITUTE GOODS OR SERVICES; LOSS OF
USE, DATA, OR PROFITS; OR BUSINESS INTERRUPTION) HOWEVER CAUSED
AND ON ANY THEORY OF LIABILITY, WHETHER IN CONTRACT, STRICT LIABILITY,
OR TORT (INCLUDING NEGLIGENCE OR OTHERWISE) ARISING IN ANY WAY OUT
OF THE USE OF THIS SOFTWARE, EVEN IF ADVISED OF THE POSSIBILITY OF
SUCH DAMAGE.



\end{document}

